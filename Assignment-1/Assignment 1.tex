\documentclass[journal,12pt,twocolumn]{IEEEtran}

\usepackage{setspace}
\usepackage{gensymb}
\singlespacing
\usepackage[cmex10]{amsmath}

\usepackage{amsthm}

\usepackage{mathrsfs}
\usepackage{txfonts}
\usepackage{stfloats}
\usepackage{bm}
\usepackage{cite}
\usepackage{cases}
\usepackage{subfig}

\usepackage{longtable}
\usepackage{multirow}

\usepackage{enumitem}
\usepackage{mathtools}
\usepackage{steinmetz}
\usepackage{tikz}
\usepackage{circuitikz}
\usepackage{verbatim}
\usepackage{tfrupee}
\usepackage[breaklinks=true]{hyperref}
\usepackage{graphicx}
\usepackage{tkz-euclide}

\usetikzlibrary{calc,math}
\usepackage{listings}
    \usepackage{color}                                            %%
    \usepackage{array}                                            %%
    \usepackage{longtable}                                        %%
    \usepackage{calc}                                             %%
    \usepackage{multirow}                                         %%
    \usepackage{hhline}                                           %%
    \usepackage{ifthen}                                           %%
    \usepackage{lscape}     
\usepackage{multicol}
\usepackage{chngcntr}

\DeclareMathOperator*{\Res}{Res}

\renewcommand\thesection{\arabic{section}}
\renewcommand\thesubsection{\thesection.\arabic{subsection}}
\renewcommand\thesubsubsection{\thesubsection.\arabic{subsubsection}}

\renewcommand\thesectiondis{\arabic{section}}
\renewcommand\thesubsectiondis{\thesectiondis.\arabic{subsection}}
\renewcommand\thesubsubsectiondis{\thesubsectiondis.\arabic{subsubsection}}


\hyphenation{op-tical net-works semi-conduc-tor}
\def\inputGnumericTable{}                                 %%

\lstset{
%language=C,
frame=single, 
breaklines=true,
columns=fullflexible
}
\begin{document}

\newcommand{\BEQA}{\begin{eqnarray}}
\newcommand{\EEQA}{\end{eqnarray}}
\newcommand{\define}{\stackrel{\triangle}{=}}
\bibliographystyle{IEEEtran}
\raggedbottom
\setlength{\parindent}{0pt}
\providecommand{\mbf}{\mathbf}
\providecommand{\pr}[1]{\ensuremath{\Pr\left(#1\right)}}
\providecommand{\qfunc}[1]{\ensuremath{Q\left(#1\right)}}
\providecommand{\sbrak}[1]{\ensuremath{{}\left[#1\right]}}
\providecommand{\lsbrak}[1]{\ensuremath{{}\left[#1\right.}}
\providecommand{\rsbrak}[1]{\ensuremath{{}\left.#1\right]}}
\providecommand{\brak}[1]{\ensuremath{\left(#1\right)}}
\providecommand{\lbrak}[1]{\ensuremath{\left(#1\right.}}
\providecommand{\rbrak}[1]{\ensuremath{\left.#1\right)}}
\providecommand{\cbrak}[1]{\ensuremath{\left\{#1\right\}}}
\providecommand{\lcbrak}[1]{\ensuremath{\left\{#1\right.}}
\providecommand{\rcbrak}[1]{\ensuremath{\left.#1\right\}}}
\theoremstyle{remark}
\newtheorem{rem}{Remark}
\newcommand{\sgn}{\mathop{\mathrm{sgn}}}
\providecommand{\abs}[1]{\vert#1\vert}
\providecommand{\res}[1]{\Res\displaylimits_{#1}} 
\providecommand{\norm}[1]{\lVert#1\rVert}
%\providecommand{\norm}[1]{\lVert#1\rVert}
\providecommand{\mtx}[1]{\mathbf{#1}}
\providecommand{\mean}[1]{E[ #1 ]}
\providecommand{\fourier}{\overset{\mathcal{F}}{ \rightleftharpoons}}
%\providecommand{\hilbert}{\overset{\mathcal{H}}{ \rightleftharpoons}}
\providecommand{\system}{\overset{\mathcal{H}}{ \longleftrightarrow}}
	%\newcommand{\solution}[2]{\textbf{Solution:}{#1}}
\newcommand{\solution}{\noindent \textbf{Solution: }}
\newcommand{\cosec}{\,\text{cosec}\,}
\providecommand{\dec}[2]{\ensuremath{\overset{#1}{\underset{#2}{\gtrless}}}}
\newcommand{\myvec}[1]{\ensuremath{\begin{pmatrix}#1\end{pmatrix}}}
\newcommand{\mydet}[1]{\ensuremath{\begin{vmatrix}#1\end{vmatrix}}}
\numberwithin{equation}{subsection}
\makeatletter
\@addtoreset{figure}{problem}
\makeatother
\let\StandardTheFigure\thefigure
\let\vec\mathbf
\renewcommand{\thefigure}{\theproblem}
\def\putbox#1#2#3{\makebox[0in][l]{\makebox[#1][l]{}\raisebox{\baselineskip}[0in][0in]{\raisebox{#2}[0in][0in]{#3}}}}
     \def\rightbox#1{\makebox[0in][r]{#1}}
     \def\centbox#1{\makebox[0in]{#1}}
     \def\topbox#1{\raisebox{-\baselineskip}[0in][0in]{#1}}
     \def\midbox#1{\raisebox{-0.5\baselineskip}[0in][0in]{#1}}
\vspace{3cm}
\title{Assignment 1}
\author{CS20BTECH11028}
\maketitle
\newpage
\bigskip
\renewcommand{\thefigure}{\theenumi}
\renewcommand{\thetable}{\theenumi}
Download all python codes from 
\begin{lstlisting}
https://github.com/Harsha24112002/AI1103/tree/main/Assignment-1/codes
\end{lstlisting}
%
and latex-tikz codes from 
%
\begin{lstlisting}
    https://github.com/Harsha24112002/AI1103/tree/main/Assignment-1
\end{lstlisting}
\section*{Question(2.13):}
Let A be the event that the sum of the numbers appearing is 6 when a die is thrown twice.\\
Let B be the event such that the number 4 appears atleast once in the two throws.\\
We need the conditional probability of event B given that A has occurred.\\ 
\begin{equation}
    \Pr{(B|A)}=\frac{\Pr{(AB)}}{\Pr{(A)}}
\end{equation}
Let $X_i \in \{1,2,3,4,5,6\},i = 1,2.$ be a random variable representing the outcome for each die.\\
The probability that A occur is same as the probability that $X_1 +X_2 =6$.\\
\begin{equation*}
\begin{split}
    \Pr{(X_1+X_2=6)}&=\Pr{(X_1=6-X_2)}\\
    &=\sum_{k}\Pr{(X_1=6-k)}\Pr{(X_2=k)}\\
\end{split}
\end{equation*}
As $1\leq X_1,X_2 \leq 6$, the equation simplifies to,\\
\begin{equation*}
        \Rightarrow \sum_{k}\Pr{(X_1=6-k)}\Pr{(X_2=k)} 
\end{equation*}
\begin{equation*}
    =\sum_{k=1}^{5}\Pr{(X_1=6-k)}\Pr{(X_2=k)}
\end{equation*}
\begin{equation*}
    =(\frac{1}{6})(\frac{1}{6})\sum_{k=1}^{5} (1)=\frac{5}{36}.
\end{equation*}
(As the probability is 1/6 for $X_1,X_2 \in \{1,2,3,4,5,6\}$)

Hence,\\
\begin{equation}
  \Rightarrow  \Pr{(A)}=\frac{5}{36}\\
\end{equation}
The event AB is such that the sum should be six with atleast one 4.
Therefore the other number must be 2.\\
There are only two possible cases \{4,2\},\{2,4\} out of 36 possible cases.\\
Hence,\\
\begin{equation}
    \Pr{(AB)}=\frac{2}{36}.    
\end{equation}
Substituting equations (0.0.2),(0.0.3) in (0.0.1) , we get\\
\begin{equation}
\begin{split}
\Pr(B|A)&=\frac{\frac{2}{36}}{\frac{5}{36}}\\
&=\frac{2}{5}.
\end{split}
\end{equation}
Hence the probability of occurring atleast one 4 when the sum of the numbers is 6 when a die is thrown twice is $\frac{2}{5}$. 

\end{document}
